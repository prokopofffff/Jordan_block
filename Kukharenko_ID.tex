\documentclass[a4paper,12pt]{article}
\usepackage[utf8]{inputenc}
\usepackage[english,russian]{babel}
\usepackage[T2A]{fontenc}

\usepackage[
  a4paper, mag=1000, includefoot,
  left=1.1cm, right=1.1cm, top=1.2cm, bottom=1.2cm, headsep=0.8cm, footskip=0.8cm
]{geometry}

\usepackage{amsmath}
\usepackage{amssymb}
\usepackage{times}
\usepackage{mathptmx}

\IfFileExists{pscyr.sty}
{
  \usepackage{pscyr}
  \def\rmdefault{ftm}
  \def\sfdefault{ftx}
  \def\ttdefault{fer}
  \DeclareMathAlphabet{\mathbf}{OT1}{ftm}{bx}{it} % bx/it or bx/m
}

\mathsurround=0.1em
\clubpenalty=1000%
\widowpenalty=1000%
\brokenpenalty=2000%
\frenchspacing%
\tolerance=2500%
\hbadness=1500%
\vbadness=1500%
\doublehyphendemerits=50000%
\finalhyphendemerits=25000%
\adjdemerits=50000%


\begin{document}

\author{Кухаренко Иван}
\title {Метод Жордана решения линейной системы}
\date{\today}
\maketitle

\section{Хранение и извлечение}

Для реализации блочного алгоритма Жордана для решения линейной системы уравнений были прописаны функции \texttt{set\_block} и \texttt{get\_block}.

\begin{enumerate}
    \item \texttt{double *A} - исходная матрица
    \item \texttt{double *block }- блок
    \item \texttt{int n} - размерность исходной матрицы
    \item \texttt{int m} - размерность блока
    \item \texttt{int i} - номер блочной строки
    \item \texttt{int j} - номер блочного столбца
\end{enumerate}

\begin{verbatim}
void set_block (double *A, double *block, size_t n, size_t m, size_t i, size_t j) {
    size_t k = n / m;
    size_t l = n % m;
    size_t w = 1, h = 1;
    if(j < k){
        w = m;
        h = m;
    }
    double *corner = A + i * n * m + j * m;
    for(size_t x = 0; x < h; x++){
        for(size_t y = 0; y < w; y++){
            corner[x * n + y] = block[x * m + y];
        }
    }
}

void get_block (double *A, double *block, size_t n, size_t m, size_t i, size_t j){
    size_t k = n / m;
    size_t l = n % m;
    size_t w = 1, h = 1;
    if(j < k){
        w = m;
        h = m;
    }
    double *corner = A + i * n * m + j * m;
    for(size_t x = 0; x < h; x++){
        for(size_t y = 0; y < w; y++){
            block[x * m + y] = corner[x * n + y];
        }
    }
}

\end{verbatim}

\section{Описание алгоритма}

Решаем линейную систему $Ax=b$ с матрицей
$$A=
   \begin{pmatrix}
     a_{1,1}& a_{1,2} &\ldots & a_{1,n}\\
     a_{2,1}& a_{2,2} &\ldots & a_{2,n}\\
     \vdots& \vdots &\ddots & \vdots\\
     a_{n,1}& a_{n,2} &\ldots & a_{n,n}
    \end{pmatrix}
$$

Припишем справа столбец свободных членов:

$$A=
   \begin{pmatrix}
     a_{1,1}& a_{1,2} &\ldots & a_{1,n} & \mid & b_1\\
     a_{2,1}& a_{2,2} &\ldots & a_{2,n} & \mid & b_2\\
     \vdots& \vdots &\ddots & \vdots\\
     a_{n,1}& a_{n,2} &\ldots & a_{n,n} & \mid & b_n
    \end{pmatrix}
$$

Заменим элементы матрицы на блоки:

$$
A=
   \begin{pmatrix}
     A^{m \times m}_{1,1}& A^{m \times m}_{1,2} &\ldots & A^{m \times m}_{1,k} & A^{m \times l}_{1,k+1} & \mid & B^{m \times 1}_1\\
     A^{m \times m}_{2,1}& A^{m \times m}_{2,2} &\ldots & A^{m \times l}_{2,k} & A^{m \times l}_{2,k+1} & \mid & B^{m \times 1}_2\\
     \vdots& \vdots &\ddots & \vdots & \vdots & \mid  & \vdots\\
     A^{m \times m}_{k,1}& A^{m \times m}_{k,2} &\ldots & A^{m \times l}_{k,k} & A^{m \times l}_{k,k+1} & \mid & B^{m \times 1}_k\\
     A^{l \times m}_{k+1,1} & A^{l \times m}_{k+1,2} & \ldots & A^{l \times m}_{k+1,k} & A^{l \times l}_{k+1,k+1} & \mid & B^{l \times 1}_{k+1}
    \end{pmatrix}
$$

Рассмотрим первый блочный столбец. Найдем первый обратимый матричный блок. Если обратной матрицы нет ни у одного матричного блока первого столбца, то система не имеет решений.Пусть это будет матричный блок $A^{m \times m}_{s,1}$ с обратной матрицей $C^{m \times m}_1$. Умножим все блоки $s$-ой блочной строки на $C^{m \times m}_1$:
$$
    A^{m \times m}_{s,j} = C^{m \times m}_1A^{m \times m}_{1,j}, \quad j=1,\dots,s-1,s+1,\dots,k, \quad A^{m \times l}_{s, k+1} = C^{m \times m}_1A^{m \times l}_{s,k+1},
$$
$$
    B^{m \times 1}_s = C^{m \times m}_1B^{m \times 1}_s,
$$

Далее необходимо занулить все матричные блоки первого блочного столбца с 1 по $k+1$ строку (кроме $s$-ой) по формулам:
$$
    A^{m \times m}_{i,j} = A^{m \times m}_{i,j} - A^{m \times m}_{i,1}A^{m \times m}_{s,j}, \quad i,j=1,\dots,s-1,s+1,\dots,k,
$$
$$
    A^{m \times l}_{i, k+1} = A^{m \times l}_{i, k+1} - A^{m \times m}_{i, 1}A^{m \times l}_{s, k+1}, \quad i=1,\dots,s-1,s+1,\dots,k,
$$
$$
    A^{l \times m}_{k+1, j} = A^{l \times m}_{k+1, j} - A^{l \times m}_{k+1, 1}A^{m \times m}_{s, j}, \quad j=1,\dots,s-1,s+1,\dots,k,
$$
$$
    A^{l \times l}_{k+1, k+1} = A^{l \times l}_{k+1, k+1} - A^{l \times m}_{k+1, s}A^{m \times l}_{s, k+1},
$$
$$
    B^{m \times 1}_i = B^{m \times 1}_i - A^{m \times m}_{i,1}B^{m \times 1}_s, \quad i=1,\dots,s-1,s+1,\dots,k,
$$
$$
    B^{l \times 1}_{k+1} = B^{l \times 1}_{k+1} - A^{l \times m}_{k+1, 1}B^{m \times 1}_s,
$$
$$
  A_{i,1}=0, \quad i=1,\dots,s-1,s+1,\dots,k+1,  
$$

После 1-ого шага исходная матрица имеет вид:

$$
    A=
   \begin{pmatrix}
     0 & A^{m \times m}_{1,2} &\ldots & A^{m \times m}_{1,k} & A^{m \times l}_{1,k+1} & \mid & B^{m \times 1}_1\\
     \vdots& \vdots &\ddots & \vdots & \vdots & \mid  & \vdots\\
     E^{m \times m} & A^{m \times m}_{s,2} &\ldots & A^{m \times m}_{s,k} & A^{m \times l}_{s,k+1} & \mid & B^{m \times 1}_s\\
     \vdots& \vdots &\ddots & \vdots & \vdots & \mid  & \vdots\\
    0 & A^{m \times m}_{k,2} &\ldots & A^{m \times m}_{k,k} & A^{m \times l}_{k,k+1} & \mid & B^{m \times 1}_k\\
     0 & A^{l \times m}_{k+1,2} & \ldots & A^{l \times m}_{k+1,k} & A^{l \times l}_{k+1,k+1} & \mid & B^{l \times 1}_{k+1}
    \end{pmatrix}
$$
На $p$-ом шаге ($p=2,\dots,k$) будем рассматривать $p$-ый блочный столбец и искать в нем обратимый матричный блок и его обратную матрицу $C^{m \times m}_p$, при этом элементы с номером столбцов с 1 по $p-1$ должны быть вырождены. Пусть такой матричный блок стоит в $q$-ой строке. Будем пользоваться следующими формулами:
$$
    A^{m \times m}_{q,j} = C^{m \times m}_pA^{m \times m}_{q,j}, \quad j=p+1,\dots,k, \quad A^{m \times l}_{q, k+1} = C^{m \times m}_pA^{m \times l}_{q,k+1},
$$
$$
    B^{m \times 1}_q = C^{m \times m}_pB^{m \times 1}_q,
$$
$$
    A^{m \times m}_{i,j} = A^{m \times m}_{i,j} - A^{m \times m}_{i,p}A^{m \times m}_{q,j},
$$
$$
    i=1,\dots,q-1,q+1,\dots,k, \; j=p+1,\dots,k,
$$
$$
    A^{m \times l}_{i, k+1} = A^{m \times l}_{i, k+1} - A^{m \times m}_{i, p}A^{m \times l}_{q, k+1}, \quad i=1,\dots,q-1,q+1,\dots,k,
$$
$$
    A^{l \times m}_{k+1, j} = A^{l \times m}_{k+1, j} - A^{l \times m}_{k+1, p}A^{m \times m}_{q, j}, \quad j=p+1,\dots,k,
$$
$$
    A^{l \times l}_{k+1, k+1} = A^{l \times l}_{k+1, k+1} - A^{l \times m}_{k+1, p}A^{m \times l}_{p, k+1},
$$
$$
    B^{m \times 1}_i = B^{m \times 1}_i - A^{m \times m}_{i,p}B^{m \times 1}_q, \quad i=1,\dots,q-1,q+1,\dots,k,
$$
$$
    B^{l \times 1}_{k+1} = B^{l \times 1}_{k+1} - A^{l \times m}_{k+1, p}B^{m \times 1}_q,
$$
$$
    A_{i,p}=0, \quad i=1,\dots,q-1,q+1,\dots,k+1,
$$
Для $k+1$ блочного столбца (если он существует) рассмотрим строку, у которой на местах с 1 по $k$ блочные матрицы вырожденые. Пусть это будет $t$-ая строка. Если элемент в этой строке в $k+1$-ом столбце вырожденый, а свободный член - нет, то решений у системы нет. Если этот матричный блок невырожденный, то найдем обратную матрицу $C^{l \times l}_{k+1}$. Далее воспользуемся следующими формулами:
$$
    A^{l \times l}_{t,k+1}=C^{l \times l}_{k+1}A^{l \times l}_{t,k+1},
$$
$$
    B^{l \times 1}_{t} = C^{l \times l}_{t}B^{l \times 1}_{t},
$$
$$
    B^{l \times 1}_i = B^{l \times 1}_i - A^{m \times l}_{i,k+1}B^{l \times 1}_{t}, \quad i=1,\dots,k,
$$
$$
    A^{m \times l}_{i,k+1}=0, \quad i=1,\dots,t-1,t+1,\dots,k+1,
$$
Тогда исходная матрица будет иметь вид с точностью до перестановок строк:
$$
    A=
   \begin{pmatrix}
     0 & E^{m \times m} &\ldots & 0 & 0 & \mid & B^{m \times 1}_1\\
     0 & 0 &\ldots & 0 & E^{l \times l} & \mid & B^{m \times 1}_2\\
     \vdots& \vdots &\ddots & \vdots & \vdots & \mid  & \vdots\\
    0 & 0 &\ldots & E^{m \times m} & 0 & \mid & B^{m \times 1}_k\\
     E^{l \times l} & 0 & \ldots & 0 & 0 & \mid & B^{l \times 1}_{k+1}
    \end{pmatrix}
$$
Тогда решением исходной системы будет вектор-столбец $B$.

\section{Сложность алгоритма}
Воспользуемся фактом, что неблочный алгоритм Гаусса для поиска обратной матрицы имеет сложность $\frac{8n^3}{3}+O(n^2)$,  умножение двух матриц $2n^3-n^2$, вычитание двух матриц $n^2$, умножение матрицы и вектора $n^2$. Пусть также $k=\frac{n}{m}$.
\begin{enumerate}
    \item Для каждого блочного столбца необходимо найти обратимую матрицу с наименьшей нормой: 
$$
    (\frac{8n^3}{3}+O(n^2))\sum^k_{j=1}j=(\frac{8n^3}{3}+O(n^2))\frac{k(k+1)}{2}=(\frac{8n^3}{3}+O(n^2))\frac{\frac{n}{m}(\frac{n}{m}+1)}{2}=
$$
$$
=\frac{4n^2m}{3}+\frac{4nm^2}{3}+O(n^2+nm)
$$
    \item Для каждой блочной строки необходимо произвести определенное количество умножений на обратную матрицу, сложность будет: 
$$
    (2m^3-m^2)\sum^{k-1}_{i=1}i=(2m^3-m^2)\frac{k(k-1)}{2}=(2m^3-m^2)\frac{\frac{n}{m}(\frac{n}{m}-1)}{2}=
$$
$$
=n^2m-nm^2+O(n^2+nm)
$$
    \item При занулении по формулам необходимо произвести умножения, сложность будет: 
$$
    (2m^3-m^2)(k-1)\sum^{k-1}_{j=1}j=(2m^3-m^2)\frac{k(k-1)^2}{2}=(2m^3-m^2)\frac{(\frac{n}{m})^2(\frac{n}{m}-1)}{2}=
$$
$$
    =n^3-2n^2m+nm^2+O(n^2+nm)
$$
    \item   При занулении по формулам необходимо произвести вычитания, сложность будет:
$$
    m^2\sum^{k-1}_{i=1}i^2=\frac{n^3}{3m}-\frac{n^2}{6}-\frac{nm}{6}=O(n^2+nm)
$$
\item Сложность операции умножения на обратную матрицу свободных членов:
$$
    m^2k=O(n^2)
$$
\item При занулении по формулам необходимо произвести умножения свободных членов, сложность будет:
$$
    m^2\sum^{k-1}_{i=1}i=O(n^2)
$$
\item При занулении по формулам необходимо произвести вычитания свободных членов, сложность будет:
$$
    m\sum^{k-1}_{i=1}i=\frac{n(\frac{n}{m})-1}{2}=O(n)
$$
\item При занулении верхних строчек необходимо произвести умножения, сложность будет:
$$
    (2m^3-m^2)\sum^{k-1}_{i=1}i^2=\frac{2n^3}{3}-\frac{n^2m}{3}-\frac{nm^2}{3}+O(n^2+nm)
$$
\item При занулении верхних строчек необходимо произвести вычитания, сложность будет:
$$
    m^2\sum^{k-1}_{i=1}i^2=O(n^2+nm)
$$
\end{enumerate}
Общая сложность алгоритма:
$$
    S(n,m)=\frac{4n^3}{3}+\frac{5n^2m}{3}-\frac{nm^2}{3}+O(n^2+nm)
$$
\section{Сложность алгоритма в краевых случаях}
$$
    S(n,1)=\frac{4n^3}{3}+O(n^2)
$$
$$
    S(n,n)=\frac{8n^3}{3}+O(n^2)
$$
\end{document}
